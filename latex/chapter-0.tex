% !Mode:: "TeX:UTF-8"
\documentclass{book}
\input{en_preamble.tex}
\input{xecjk_preamble.tex}
\begin{document}
\title{数值代数讲义}
\maketitle

\chapter{引言}

\section{课程主要内容}
\begin{itemize}
    \item Python 数值计算基础
    \item 线性方程组的直接解法 (Gauss 消去法 / LU 分解)
    \item 线性最小二乘问题的数值算法
    \item 非对称矩阵的特征值计算
    \item 对称矩阵特征值计算与奇异值分解
    \item 线性方程组迭代算法 (大规模, 稀疏, 特殊结构)
    \item 特征值问题的迭代算法 (大规模, 稀疏, 部分特征值和特征向量)
\end{itemize}

\section{主要参考资料}

\begin{itemize}
    \item G. Golub and C. F. van Loan, “Matrix Computations (4th),” Johns Hopkins University Press, 2013.
中文版:《矩阵计算》(第三版), 袁亚湘等译, 科学出版社, 2001.
\item J. W. Demmel, “Applied Numerical Linear Algebra,” SIAM, 1997.
中文版:《应用数值线性代数》, 王国荣译, 人民邮电出版社, 2007.
\item L. N. Trefethen and D. Bau, III, “Numerical Linear Algebra,” SIAM, 1997.
中文版:《数值线性代数》, 陆金甫等译, 人民邮电出版社, 2006.
\item 徐树方, “矩阵计算的理论与方法,” 北京大学出版社, 1995.
\item 曹志浩, “数值线性代数,” 复旦大学出版社, 1996.
\end{itemize}

\section{二十世纪最优秀的十大算法}
The Best of the 20th Century: Editors Name Top 10 Algorithms, SIAM News, Volume 33, Number 4, 2000.
\begin{enumerate}
    \item Monte Carlo method (1946)
    \item Simplex Method for Linear Programming (1947)
    \item {\color{blue} Krylov Subspace Iteration Methods (1950)}
    \item {\color{blue} The Decompositional Approach to Matrix Computations
        (1951)}
    \item The Fortran Optimizing Compiler (1957)
    \item {\color{blue} QR Algorithm for Computing Eigenvalues (1959-61)}
    \item Quicksort Algorithm for Sorting (1962)
    \item Fast Fourier Transform (1965)
    \item Integer Relation Detection Algorithm (1977)
    \item Fast Multipole Method (1987)
\end{enumerate}
(蓝色: 属于数值线性代数; 下划线: 与数值线性代数密切相关.)


\cite{shi2000}
\bibliographystyle{abbrv}
\bibliography{ref}
\end{document}
